\newpage

\addcontentsline{toc}{chapter}{Eidesstattliche Erklärung}					% IN Inhaltsverzeichnis
\vspace*{\fill}																% Vertikal Zentrieren

{

\begin{center}
	\huge {\bf Eidesstattliche Erklärung}
	\label{cha:eidesstattlicheErklaerung}
\end{center}

\vspace{2cm}

Gemäß § 5 (3) der „Studien- und Prüfungsordnung DHBW Technik“ vom 22. September 2011.\\
Ich habe die vorliegende Arbeit selbstständig verfasst und keine anderen als die angegebenen 
Quellen und Hilfsmittel verwendet. \\
\vspace{2cm}


\begin{table}[h]
%	\centering
		\begin{tabular}{ll}
\rule{6.5cm}{0.3pt} & \rule{6.5cm}{0.3pt}\\
Ort, Datum  & Unterschrift\\
&\\
&\\
\rule{6.5cm}{0.3pt} & \rule{6.5cm}{0.3pt}\\
Ort, Datum  & Unterschrift\\
		\end{tabular}
\end{table}

}

\vspace*{\fill}
\newpage