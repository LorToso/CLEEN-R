\chapter{Problemstellung}
\label{cha:Problemstellung}

Ziel dieser Studienarbeit ist die praktische Anwendung gelernter Kenntnisse in Hard- und Software. Durch die Benutzung zweier verschiedener getrennter Module sind sowohl Kenntnisse in der Programmierung, als auch der Prozessautomatisierung und in der Entwicklung verteilter Systeme erforderlich.

Konkrete Aufgabenstellung ist es einen kameragestützten Roboter zu entwickeln, der autonom Gegenstände in einem Raum erkennt und in definierte Zielzonen transportiert. Der Prozess kann hierbei in drei Teilprozesse unterteilt werden. 

Erstens muss der Roboter mit Hilfe von bekannten Verfahren der Bildverarbeitung Objekte auf Grund ihrer physikalischen Beschaffenheit, beispielsweise ihrer Größe, ihrer Form und ihrer Farbe, erkennen und zielgerichtet anfahren.

Zweitens soll sich der Roboter gegenüber des Gegenstands optimal positionieren und diesen mit Hilfe eines mechanischen Greifarms aufnehmen.

Zuletzt muss der Roboter den aufgenommenen Gegenstand kategorisieren, nach der entsprechenden Zielzone suchen diese Anfahren und den Gegenstand ablegen. Die Erkennung der Zielzone kann dabei durch Markierungen an Wänden und Böden des Raumen erfolgen.

Als Bewertungskriterien dienen hierbei beispielsweise ob der Roboter alle Gegenstände erfolgreich erkennt, diese korrekt kategorisiert und in korrekte Zielzonen bewegt, sowie die Zeit in der dies geschieht.