\chapter{Einleitung}
\label{cha:einleitung}

Im Rahmen der Studienarbeit des fünften und sechsten Semesters für die Prüfung zum Bachelor of Engineering, stellt diese Arbeit eine Dokumentation zur Entwicklung eines kameragestützten Roboters dar. Ziel der Arbeit ist es mit Hilfe eines Android-Smartphones und eines LEGO Mindstorm NXT-Kits einen Roboter zu entwerfen, der Gegenstände in einem Raum erkennt, anfährt und in eine vordefinierte Zielzone transportiert. Hierfür werden diverse Methoden der Bildverarbeitung eingesetzt, welche unter der Verwendung der OpenCV-Bibliothek \cite{opencv_library} implementiert werden. Das Projekt wurde \glqq CLEEN-R\grqq\ getauft. Der Name stellt hierbei ein Akronym für \glqq \textbf{C}leaning and \textbf{L}ifting \textbf{E}nviroment \textbf{E}valuating \textbf{N}XT - \textbf{R}obot\grqq\ und damit eine Anspielung an den Disney-Film \glqq WALL-E\grqq\ dar.

Das nachfolgende Kapitel beschreibt die \hyperref[cha:Problemstellung]{Problemstellung} und erklärt eine Grundproblematik der Zusammenarbeit der beiden Hardwaremodule. Kapitel \chapNoToTextRef{cha:Materials} beinhaltet genaue Daten zu den Hardwaremodulen und deren Zusammenspiel. Das Kapitel \hyperref[cha:robot]{Hardwareumsetzung} beinhaltet gesondert den Aufbau des Roboters, sowie die Ansteuerung einzelner Aktoren und Sensoren. Notwendige Grundlagen der Bildverarbeitung, sowie genaue Algorithmen zur Objekterkennung und genauer Aufbau der Applikation werden im darauf folgenden Kapitel \hyperref[cha:Software]{Softwareumsetzung} beschrieben. Den Kern der Arbeit stellt das Kapitel \hyperref[cha:Workloop]{Arbeitsablauf und Problematiken} dar. Es beinhaltet den genauen Arbeitszyklus des Roboters und beschreibt Problematiken, die während der Entwicklung auftraten, sowie Designentscheidungen zur deren Lösung. Kapitel \ref{cha:Tests} beschreibt durchgeführte Tests sowohl unter speziell präparierten, als auch unter Realbedingungen. Das \hyperref[cha:Fazit]{abschließende Kapitel} ist ein letztes Fazit, welches einen Überblick über die gesamte Arbeit bildet und einen Ausblick über aufbauende Arbeiten liefert.



