\chapter{Einleitung}
\label{cha:einleitung}

Im Rahmen der Studienarbeit des fünften und sechsten Semesters der Prüfung zum Bachelor of Engineering, stellt diese Arbeit eine Dokumentation zur Entwicklung eines kameragestützten Roboters dar. Ziel der Arbeit ist es mit Hilfe eines Android-Smartphones und eines LEGO Mindstorm NXT-Kits einen Roboter zu entwerfen, der Gegenstände in einem Raum erkennt, anfährt und in eine vordefinierte Zielzone transportiert. Hierfür werden diverse Methoden der Bildverarbeitung eingesetzt, welche unter der Verwendung der OpenCV-Library \cite{opencv_library} implementiert werden.

Das nachfolgende Kapitel beschreibt die \hyperref[cha:Problemstellung]{Problemstellung} und erklärt eine Grundproblematik der Zusammenarbeit der beiden Hardwaremodule. Kapitel \chapNoToTextRef{cha:Materials} beinhaltet genaue Daten zu den Hardwaremodulen und deren Zusammenspiel, sowie die notwendigen Grundlagen der Bildverarbeitung, die für spätere Methoden genutzt werden. Die beiden darauf folgenden Kapitel gehen gesondert auf den genauen Aufbau des Roboters samt Konstruktionsplänen, sowie Details zur Softwareimplementierung der Objekterkennung und der Roboteransteuerung ein. Kapitel \ref{cha:tests} beschreibt durchgeführte Tests sowohl unter speziell präparierten Bedingungen, als auch Realbedingungen. Das \hyperref[cha:Fazit]{abschließende Kapitel} ist ein letztes Fazit, welches einen Überblick über die gesamte Arbeit bildet.



