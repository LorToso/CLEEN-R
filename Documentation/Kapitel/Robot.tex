\chapter{Hardwareumsetzung}
\label{cha:robot}
\section{Anforderungen an den NXT-Roboter}

Der Roboter hat einige hardwareseitige Anforderungen. Das System muss sich frei im Raum bewegen können, sowie eine Möglichkeit bieten mit Gegenständen zu interagieren. Diese müssen aufgehoben, transportiert und zielgerichtet abgelegt werden können. Es ist zusätzlich wichtig, dass die Gegenstände beim Ablegen sicher an ihrer Position verharren, damit sie sich nicht aus einer möglichen Zielzone heraus bewegen. Wichtiges Kriterium ist zusätzlich eine sichere Halterung für das Kameramodul, welches in Kapitel \ref{sec:Kamera} näher beschrieben ist.

\section{Entwurf des Roboters}

Da das LEGO Mindstorm NXT Kit häufig zur Realisierung kleinerer Roboterprojekte verwendet wird, hat LEGO eine Datenbank an möglichen Bauplänen bereitgestellt \cite{building_instructions}. Nach einiger Recherche und Durchsicht diverser Bauanleitungen für verschiedenste Anwendungsbereiche wurde sich für einen von LEGO bereitgestellten Aufbau den entschieden. Abbildung \ref{fig:standardRoboter} zeigt den Roboter, wie LEGO ihn bereitstellt.

\begin{figure}[h]
\centering
\todo{Bild von Standard Roboter einfügen}
\caption{Von Lego bereitgestellter Roboteraufbau}
\label{fig:standardRoboter}
\end{figure}


Da der Aufbau jedoch nicht allen Anforderungen genüge tut, wurde dieser abgeändert. Der von LEGO bereitgestellte Bauplan benutzt einen Schall- und einen Farbsensor. Diese wurden nicht benötigt und daher entfernt. An ihrer Stelle wurde stattdessen eine Halterung für das in Kapitel \ref{sec:Kamera} beschriebene Kameramodul in Form eines Smartphones eingebaut. Um dem Gewicht des Smartphones entgegenzuwirken musste der in Abschnitt \ref{subsec:Brick} beschriebene NXT-Stein etwas nach hinten verlagert werden. Abbildung \ref{fig:unserRoboter} zeigt den fertig modifizierten Roboter.

\begin{figure}[h]
\centering
\todo{Bild des fertig modifizierten Roboters}
\caption{Bild des fertig modifizierten Roboters}
\label{fig:unserRoboter}
\end{figure}

\subsection{NXT-Stein}
\label{subsec:Brick}
\begin{figure}[h]
\centering
\includegraphics[width=0.5\textwidth]{Bilder/Robot/nxt_brick}
\caption{NXT-Stein}
\label{fig:nxtBrick}
\end{figure}

Der NXT-Stein bildet die Recheneinheit und damit die Hauptkomponente des Robotersystems \cite{ranganathan2008use}. Er besitzt drei Ausgänge für Motoren an der Oberseite, über die gleichzeitig die Rotationssensoren in den Servo-Motoren ausgelesen werden.

Auf der Unterseite befinden sich vier Eingänge für verschiedene Sensoren, die je nach Anwendungszweck über Flachbandkabel bestückt werden können, etwa ein Licht-, Schall-, Tastsensor oder beliebige Kombinationen daraus. Dies macht das NXT-System zu einem sehr flexiblem da einfach konfigurierbaren Robotersystem.

Über einen USB-Anschluss oben wird der NXT mit dem PC verbunden um ihn mit Programmen zu versorgen. Hierfür wird von LEGO eine graphische Entwicklungsumgebung bereitgestellt um auch Neulingen den Einstieg in die Roboter-Programmierung zu vereinfachen und ihnen die Möglichkeit zu bieten schon innerhalb weniger Minuten einen funktionierenden Prototypen zu konstruieren. Im Fall des CLEEN-R-Aufbaus wird jedoch, wie in Abschnitt \ref{sec:RoboterSteuerung} geschildert, lediglich die interne Bluetooth-Verbindung genutzt.

Auf der Vorderseite befindet sich ein $100\times 64$ Pixel auflösendes binäres LCD-Display über das Einstellungen getätigt oder Statusmeldungen ausgegeben werden können. Auch Sound-Ausgabe über einen integrierten 8-Bit-Lautsprecher ist möglich.

\subsection{Sensoren}

LEGO stellt unterschiedliche Sensoren für den Betrieb an einem NXT-Robotersystem bereit. Die wichtigsten Sensortypen sind in den nachfolgenden Abschnitten beschrieben.

\subsubsection{Tastsensor}

\begin{figure}[h]
\centering
\includegraphics[width=\textwidth/3]{Bilder/Robot/button_sensor}
\caption{Tastsensor des NXT-Systems}
\label{fig:buttonSensor}
\end{figure}

Der berührungsempfindliche Tastsensor dient in der ursprünglichen Vorlage zur Detektion von Gegenständen im Bereich des Greifarms. Wird er betätigt, so kann der Greifarm geschlossen werden. Im CLEEN-R-Projekt wurde dieser jedoch durch einen Ultraschallsensor ersetzt, der nicht nur die unmittelbare Berührung bemerkt sondern auch ein herannahendes Objekt während der Fahrt detektiert. Der Ultraschallsensor ist in einem nachfolgenden Abschnitt beschrieben.

\subsubsection{Rotationssensoren}

Die Servomotoren des Roboters sind mit Rotationssensoren ausgestattet. Diese Sensoren erlauben es dem NXT-Roboter, die Geschwindigkeit der Motoren abhängig des Widerstands des Untergrunds zu regulieren. So werden unter anderem präzises Abbremsen und Fehlerminimierung bei der Positionsbestimmung ermöglicht.

\subsubsection{Farbsensor}

\begin{figure}[h]
\centering
\includegraphics[width=\textwidth/3]{Bilder/Robot/color_sensor}
\caption{Farbsensor des NXT-Systems}
\label{fig:colorSensor}
\end{figure}

Dem ursprünglichen Entwurf des Roboters liegt ein RGB-Farbsensor bei. Seine Aufgabe war es die Farbe des Untergrunds festzustellen. Im CLEEN-R-Projekt wurde dieser Sensor zwar entfernt, könnte jedoch bei Erweiterung des Roboters dazu genutzt werden um die Farbe des Untergrunds aus Aufnahmen des Kameramoduls zu entfernen. Dies würde die Qualität der Objekterkennung auf farbigen Untergrund deutlich erhöhen. Zusätzlich ist es denkbar die Erkennung von Zielzonen über farbige Markierungen auf dem Untergrund zu realisieren.

\subsubsection{Ultraschallsensor}
\label{subsec:Ultraschallsensor}

\begin{figure}[h]
\centering
\includegraphics[width=\textwidth/3]{Bilder/Robot/distance_sensor}
\caption{Ultraschallsensor des NXT-Systems}
\label{fig:distanceSensor}
\end{figure}

Der Ultraschallsensor dient zur Messung der Distanz vom Roboter zum nächsten soliden Objekt. Der Sensor hat eine Reichweite von $255cm$ und kann mit einer Präzision von $\pm 3cm$ Entfernungen angeben. Der Sensor wird in diesem Projekt benutzt um zu detektieren, wie weit ein Objekt vom Greifarm entfernt ist. Hat es einen gewissen Abstand erreicht, so kann der Greifarm geschlossen werden um das Objekt aufzunehmen. 

In Kapitel \ref{subsec:Entfernungsschätzung} werden alternative Möglichkeiten zur Entfernungsmessung geschildert. Der Ultraschallsensor stellt sich jedoch als geeignetstes Mittel heraus und wird daher im folgenden verwendet.

\subsection{Aktoren}

Die Aktoren des Robotersystems beschränken sich beim CLEEN-R-Projekt auf Servomotoren. Im Folgenden werden die unterschiedlichen Anwendungen der Motoren beschrieben.

\subsubsection{Antriebsmotoren}

\begin{figure}[h]
\centering
\includegraphics[width=\textwidth/3]{Bilder/Robot/motor}
\caption{Servo-Motor des NXT-Systems}
\label{fig:motor}
\end{figure}

Der NXT-Roboter verwendet zwei Servomotoren an den Seiten als Differentialantrieb. Dieser ermöglicht eine freie Fortbewegung und das Rotieren des Roboters. 

\subsubsection{Greifarm}
\label{greifarm}
Der Greifarm des Roboters wird über einen dritten Motor auf der Vorderseite bewegt. Der Motor erlaubt das freier öffnen und schließen des Greifarms und somit das mitführen von Gegenständen. 

Bei der Ansteuerung des Greifarmmotors ist zu beachten, dass dieser nicht zu schnell geschlossen werden sollte, da dies zu tragende Objekte wegstoßen könnte. Abbildung \ref{fig:Greifarm} zeigt den Greifarm in offenem und geschlossenem Zustand .

\begin{figure}[h]
\centering
\todo{Bild des Greifarms einfügen}
\caption[Greifarm des Robotersystems]{Greifarm des Robotersystems in offenem Zustand (links) und geschlossenem Zustand (rechts)}
\label{fig:Greifarm}
\end{figure}

\section{Steuerung des Roboters}
\label{sec:RoboterSteuerung}

Die Steuerung des Roboters durch das Smartphone erfolgt via Bluetooth. Der NXT-Roboter stellt hierzu als Bluetooth-Gerät das Serial Port Profile (SPP) bereit, welches das Standardinterface auf Bluetooth-Ebene darstellt und vom Großteil aller Bluetooth-Hostgeräte unterstützt wird. Es ermöglicht das simple serielle Senden und Empfangen von Bytes auf Low-Level-Ebene.

Das Kommunikationsprotokoll auf High-Level-Ebene und damit die nötigen Befehle zum Regeln der Aktoren und Auslesen der Sensoren wurde von LEGO dokumentiert und ist online erhältlich \cite{nxt_comm_protocol}.

\begin{figure}[h]
\centering
\includegraphics[width=\textwidth/2]{Bilder/Robot/bluetooth}
\caption{Bluetooth-Verbindung zwischen NXT und Nexus 5}
\label{fig:bluetooth}
\end{figure}

Um die Logik zentral zu halten wird auf dem NXT selbst kein Programm ausgeführt. Das Smartphone führt zentral alle Berechnungen durch sendet dem Roboter lediglich Befehle zur Ansteuerung der Aktoren.

Zunächst muss eine Bluetooth-Verbindung erstellt werden, wozu beide Geräte aktiviertes Bluetooth aufweisen, der Roboter zusätzlich sichtbar für das Smartphone sein müssen.

Bei Erstverbindung muss der gesuchte NXT ausgewählt werden, danach ist die Bluetooth-Adresse bekannt und die App kann ohne Benutzerinteraktion eine Verbindung mit dem NXT-Roboter aufnehmen.

Kommt eine Verbindung zustande, können seriell Byte für Byte die Kommandos an den NXT übertragen, eventuelle Antworten empfangen werden.

App-seitig übernimmt ein gesonderter Thread in der Klasse NxtTalker nach Zustandekommen einer Verbindung das Management der Daten.

Zum Bewegen der Motoren muss zunächst per Befehl pro Aktor eine Geschwindigkeit im Bereich von -100\% bis 100\% (und Parameter wie Regulierung) übergeben, zum Stoppen können alle Motoren mit einem Befehl auf Geschwindigkeit '0' gesetzt werden.

Zwei Motoren können synchronisiert werden, sodass diese gleichzeitig starten. Ansonsten würde der Zeitversatz zwischen dem Absetzen der zwei 'setze Geschwindigkeit'-Befehle bewirken, dass der Roboter vor dem geradeaus fahren kurz nur das erste Rad ansteuert und in eine Richtung abdriftet.

\section{Wahl des Kameramoduls}
\label{sec:Kamera}

Die Hauptfrage bezüglich des Kameramoduls bestand in der Wahl zwischen einem Ein- oder einem Zweikamerasystem.

Der Vorteil eines Zweikamerasystems besteht in der Möglichkeit für wesentlich bessere Orientierung im 3D-Raum, da Entfernungen mittels der beiden Differenzbilder präziser berechnet werden können.
Im Gegensatz dazu ist beim Einkamerasystem die Entfernungsberechnung auf ein 2D-Bild beschränkt und nicht annähernd so genau.

Jedoch ist der Berechnungsaufwand für das Auswerten zweier Differenzbilder ungleich höher, weshalb sich letztendlich aufgrund dieser Ungleichheit des Implementierungsaufwandes für ein Einkamerasystem entschieden werden.

Weitere Aspekte sind Auflösung und Öffnungswinkel des Kameramoduls.
Höhere Auflösung bedeutet bessere Erkennung von Gegenständen auf weitere Entfernungen; Ein größerer Öffnungswinkel heißt, dass mehr Raum in einem Bild erfasst werden kann, somit weniger Drehbewegung des Roboters in Richtung eines Objekts nötig ist, bis es erfasst und detektiert werden kann.

Ein Problem bei der Kamera des Nexus ist, das sich diese nicht zentral auf dem Rücken des Smartphones befindet, sondern nach links oben versetzt. Dies setzt voraus, dass entweder Software- oder Hardwareseitig dieser Versatz aus der Bildverarbeitung kompensiert wird.

Hier wurde der Einfachheit halber die Halterung am Roboter verschoben, wodurch aus der Sicht des Roboters die Kamera in der Mitte liegt.

\begin{figure}[h]
\centering
\includegraphics[width=\textwidth/3]{Bilder/Robot/nexus_backside}
\caption{Kameramodul des Nexus 5}
\label{fig:camera}
\end{figure}

