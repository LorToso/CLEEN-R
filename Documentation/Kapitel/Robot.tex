\chapter{Hardwareumsetzung}
\section{Entwurf des NXT-Roboters}

Die hardwareseitigen Voraussetzungen an den Roboter bestanden im Hauptsächlichen aus der freien Bewegung im Raum und dem Aufnehmen, Mitführen und Ablegen von kleinen Gegenständen in einem vordefinierten Bereich.

Nach kurzer Recherche\cite{building_instructions} und Durchsicht von Bauanleitungen für verschiedenste Anwendungsbereiche wurde sich für den Standardaufbau aus der zum Bauset zugehörigen LEGO NXT Bauanleitung entschieden.

Sie wurde lediglich um den Schall- und den Abstandssensor erleichtert; eine Halterung für das Smartphone wurde hinzugefügt.

\todo{hier Bild des Roboters einfügen}
\pagebreak

\subsection{Sensoren}

\begin{description}
\item[Tastsensor]\hfill \\
Der berührungsempfindliche Sensor vorne dient zum Detektieren von Gegenständen im Bereich des Greifarms, woraufhin dieser geschlossen werden kann.
\item[Rotationssensoren]\hfill \\
Die Rotationssensoren in den Servomotoren erlauben es dem NXT-Roboter, die Geschwindigkeit der Motoren abhängig des Widerstands (des Untergrunds) zu regulieren. So werden unter anderem präzises Abbremsen und Fehlerminimierung bei der Positionsbestimmung ermöglicht.
\end{description}

\subsection{Aktoren}

\begin{description}
\item[Antriebsmotoren]\hfill \\
Die beiden Servomotoren links und rechts des NXT-Roboters bilden differentialen Antrieb und ermöglichen freie Fortbewegung.
\item[Greifarmmotor]\hfill \\
Der dritte Motor im vorderen Teil des Roboters dient zum Öffnen und Schließen des Greifarms und so zur Mitführung von Gegenständen.
\end{description}

\section{Steuerung des Roboters}
Bluetooth Verbindung zu Smartphone

\section{Wahl des Kameramoduls}
\label{sec:Kamera}
