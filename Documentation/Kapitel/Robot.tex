\chapter{Hardwareumsetzung}
\section{Entwurf des NXT-Roboters}

Die hardwareseitigen Voraussetzungen an den Roboter bestanden im Hauptsächlichen aus der freien Bewegung im Raum und dem Aufnehmen, Mitführen und Ablegen von kleinen Gegenständen in einem vordefinierten Bereich.

Nach kurzer Recherche\cite{building_instructions} und Durchsicht von Bauanleitungen für verschiedenste Anwendungsbereiche wurde sich für den Standardaufbau aus der zum Bauset zugehörigen LEGO NXT Bauanleitung entschieden.

Sie wurde lediglich um den Schall- und den Abstandssensor erleichtert; eine Halterung für das Smartphone wurde hinzugefügt.

[HIER BILD DES ROBOTERS EINFÜGEN]

\subsection{Sensoren}

\begin{description}
\item[Tastsensor]\hfill \\
Der berührungsempfindliche Sensor vorne dient zum Detektieren von Gegenständen im Bereich des Greifarms, woraufhin dieser geschlossen werden kann.
\end{description}

\subsection{Aktorik}

\begin{description}
\item[Antriebsmotoren]\hfill \\
Die beiden Servomotoren links und rechts des NXT-Roboters bilden differentialen Antrieb und ermöglichen freie Fortbewegung.
\end{description}

\section{Steuerung des Roboters}
Bluetooth Verbindung zu Smartphone

\section{Wahl des Kameramoduls}
\label{sec:Kamera}
