\chapter{Zusammenfassung und Ausblick}
\label{cha:Fazit}

Im Rahmen dieser Studienarbeit wurde das Robotersystem CLEEN-R entwickelt. Ziel war es mit Hilfe eines LEGO\textregistered\ Mindstorm NXT-Robotersets und eines Google Nexus 5 Android Smartphone ein Robotersystem zu entwickeln, welches kameragestützt Objekte erkennen, kategorisieren und transportieren sollte. 

In den vorhergehenden Kapiteln wurde zunächst die \hyperref[cha:Materials]{konkrete Anwendung der Hardware}  beschrieben. Daraufhin ist die \hyperref[cha:robot]{Konstruktion des Roboters} geschildert. In den darauf folgenden Kapiteln sind \hyperref[cha:Software]{genaue Algorithmen zur Bildverarbeitung}, sowie der \hyperref[cha:Workloop]{allgemeine Arbeitszyklus} des Systems beschrieben. Zuletzt folgen \hyperref[cha:Tests]{Tests in geschützten Raum, sowie Realtests}.

Die Hauptaufgabe wurde erfolgreich gelöst. Wie aus Kapitel \ref{cha:Tests} ersichtlich, konnte ein Robotersystem konstruiert werden, welches erfolgreich Gegenstände \glqq aufräumen\grqq\ kann. Das System besteht aus zwei getrennten Modulen: Ein LEGO Mindstorm NXT-Roboter und ein Google Nexus 5 Android-Smartphone kommunizieren über eine Bluetooth-Schnittstelle mit einander und steuern so die Aktoren des Roboters. 

Es konnte ein Großteil der Komplexität des Systems dadurch reduziert werden, dass das Smartphone als zentrale Steuereinheit benutzt wird und das NXT-System lediglich als Vermittler zu Sensoren und Aktoren genutzt wird. Der Roboter basiert auf einem modifizierten Bauplan, der von LEGO zur Verfügung gestellt wurde. Dieser Bauplan wurde dahingehend angepasst, dass einige Sensoren entfernt und durch eine Halterung für das Smartphone ersetzt wurden.