\chapter{Zusammenfassung und Ausblick}
\label{cha:Fazit}

Im Rahmen dieser Studienarbeit wurde das Robotersystem CLEEN-R entwickelt. Ziel war es mit Hilfe eines LEGO\textregistered\ Mindstorm NXT-Robotersets und eines Google Nexus 5 Android Smartphone ein Robotersystem zu entwickeln, welches kameragestützt Objekte erkennen, kategorisieren und transportieren sollte. 

In den vorhergehenden Kapiteln wurde zunächst die \hyperref[cha:Materials]{konkrete Anwendung der Hardware}  beschrieben. Daraufhin ist die \hyperref[cha:robot]{Konstruktion des Roboters} geschildert. In den darauf folgenden Kapiteln sind \hyperref[cha:Software]{genaue Algorithmen zur Bildverarbeitung}, sowie der \hyperref[cha:Workloop]{allgemeine Arbeitszyklus} des Systems beschrieben. Zuletzt folgen \hyperref[cha:Tests]{Tests in geschützten Raum, sowie Realtests}.

\section{Zusammenfassung}

Wie aus Kapitel \ref{cha:Tests} ersichtlich, konnte ein Robotersystem konstruiert werden, welches erfolgreich Gegenstände \glqq aufräumen\grqq\ kann. Das System besteht aus zwei getrennten Modulen: Ein LEGO Mindstorm NXT-Roboter und ein Google Nexus 5 Android-Smartphone kommunizieren über eine Bluetooth-Schnittstelle mit einander und steuern so die Aktoren des Roboters. 

Es konnte ein Großteil der Komplexität des Systems dadurch reduziert werden, dass das Smartphone als zentrale Steuereinheit benutzt wird und das NXT-System lediglich als Vermittler zu Sensoren und Aktoren genutzt wird. Der Roboter basiert auf einem modifizierten Bauplan, der von LEGO zur Verfügung gestellt wurde. Dieser Bauplan wurde dahingehend angepasst, dass einige Sensoren entfernt und durch eine Halterung für das Smartphone ersetzt wurden.

Die Implementierung erfolgte mit der OpenCV-Bibliothek in einer Android-Applikation. Hierbei wurden verschiedene Verfahren der Bildverarbeitung eingesetzt um Objekte zu erkennen. Konkret wird in den Kameraaufnahmen nach Objekten mit hoher Farbsättigung gesucht. Bilder werden anschließend binarisiert und mit Hilfe von Algorithmen zur Kantenverfolgung segmentiert. 

Ist ein Objekt erkannt, so fährt der Roboter es an und nimmt es mit einem Greifarm auf. Das Ansteuern des Objekts geschieht hierbei mit einer Kombination aus Sensordaten aus einem Ultraschallsensor und der Kamera des Smartphones. Wenn das Objekt erfolgreich aufgenommen wurde, wird es auf Grund seiner Farbe und Form kategorisiert. Je nach Kategorie, wird eine andere Zielzone ausgesucht und lokalisiert.

Ist eine Zielzone bestimmt worden, transportiert der Roboter den aufgenommenen Gegenstand dort hin und legt ihn ab. Anschließend begibt er sich zurück in die Startposition und beginnt die Suche von vorne. Die Orientierung im Raum wurde dabei über einen Positionsverfolgungsansatz gelöst, welcher die gefahrene Strecke berechnet um die aktuelle Position zu bestimmen.

\section{Bewertung der Ergebnisse}

Die Hauptaufgabe wurde erfolgreich gelöst. Es wurde ein Robotersystem entwickelt, welches alle in der \hyperref[cha:Problemstellung]{Problemstellung} beschriebenen Aufgaben erfolgreich Lösen kann. 

Bei der Entwicklung traten an vielen Stellen, wie beispielsweise der in Kapitel \ref{subsec:Entfernungsschätzung} beschriebenen Entfernungsschätzung, schwierige Probleme auf, welche durch zum Teil sehr elegante Lösungswege gelöst werden konnten. Es wurden stets mehrere Möglichkeiten abgewägt, wobei sich an einigen Stellen gezeigt hat, dass zusätzliche Hardware eine elegantere Lösung darstellt. Es ist jedoch auch zu beachten, dass diese immer einen zusätzlichen Entwicklungsaufwand mit sich führt und zusätzliche Mittel in Anspruch nimmt, die bei einem kleineren Roboterprojekt oft nicht vorhanden sind. 

Wie Kapitel \ref{cha:Tests} zeigt, funktioniert die Koppelung des Smartphones als Kameramodul mit dem NXT-Roboter. Die zentrale Designentscheidung das Smartphone als Steuereinheit zu benutzen und den Roboter lediglich ausführende Tätigkeiten verrichten zu lassen hat sich als sehr gewinnbringend herausgestellt. Durch die geringe Rechenleistung des NXT-Steins wäre eine effiziente Berechnung von Algorithmen der Bildverarbeitung nicht möglich gewesen.

Über das Verhalten des Roboters lässt sich sagen, dass es noch sehr abgehackt wirkt. Bewegungen sind nicht flüssig, Drehungen oft langsam und das Positionstracking manchmal ungenau. All diese Einflüsse erwecken den Eindruck der Roboter sei unfertig. In nachfolgenden Abschnitt wird ein Ausblick über weitere Vorgehensweisen bei der Bearbeitung dieser Themen gewährt.

\section{Ausblick}

Ein Hauptproblem des aktuellen Robotersystems stellt das \glqq abgehackte\grqq\ Fahren dar. Der Roboter fährt nie eine Kurve, sondern hält an und dreht sich auf der Stelle. Dies erleichtert das Positionstracking und das Zentrieren von Objekten ohne das Risiko sie zu Überfahren, ist jedoch zeitlich ineffizient und wird von Beobachtern als \glqq unschön\grqq\ empfunden. Ein flüssigeres Verhalten kann durch den ausgefeilteren Einsatz des Ultraschallsensors erreicht werden. Hierbei kann die Geschwindigkeit des Heranfahrens an ein Objekt an die gemessene Entfernung angepasst werden.

Eine weitere Möglichkeit der Erweiterung des Systems stellt die Verwendung einer zweiten Kamera dar. Diese zweite Kamera kann zwei mögliche Anwendungen haben. Sie könnte als zusätzliches Kameramodul auf dem Roboter Tiefeninformationen aus Bildern erzeugen und so eine Kartografierung der Umgebung beginnen. Der Roboter könnte so als lernendes System Räume einstudieren und zu transportierende Gegenstände als \glqq Fremdkörper\grqq\ im Raum identifizieren. Alternativ könnte eine zweite Kamera als externer Beobachter des Systems verwendet werden. Dies würde wie in Kapitel \ref{subsec:Beobachter} beschrieben zu einer genaueren Positionserkennung beitragen und das System somit stabiler gegen Ungenauigkeiten der Rotationssensoren machen.

Insgesamt lässt sich sagen, dass noch einige offene Nahtstellen im Projekt vorhanden sind, an denen zukünftige Entwickler Erweiterungen und Verfeinerungen anbringen können. Durch die modulare Struktur der Applikation lässt sich nahezu der gesamte Code ohne große Änderungen für weitere Projekte benutzen.

\section{Reflexion des Gelernten}