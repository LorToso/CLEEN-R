\chapter*{Abstract}
\addcontentsline{toc}{chapter}{Abstract}

\textbf{Deutsch:}\\
In dieser Studienarbeit wurde ein Robotersystem entwickelt, dessen Ziel es ist kameragestützt Objekte zu erkennen. Der Roboter fährt die Gegenstände an, kategorisiert diese und transportiert sie abhängig von der Kategorisierung in eine Zielzone. Für die Realisierung wurde ein LEGO Mindstorms NXT Roboter und ein Google Nexus 5 Smartphone verwendet. Die Objekterkennung wurde mit bekannten Verfahren aus der Bildverarbeitung mit Hilfe der OpenCV-Bilbiothek implementiert. Der Roboter orientiert sich im Raum mit Hilfe einer Positionsverfolgung über die gefahrene Strecke. Unter Verwendung dieser Methoden konnte ein System entworfen werden, welches einen Gegenstand in weniger als einer Minute aufräumen kann.

\vspace{1cm}

\textbf{English:}\\
In this student research project a robotic system was developed, which has the goal to recoginize objects with an optical camera module. The robot approaches an object, categorizes it and transports it to a target location. Which target zone it is transported to depends on the categorization. The robot was built using a LEGO Mindstorms NXT kit and a Google Nexus 5 Smartphone. The object recognition algorithm was realized using methods known in image processing with the aid of the OpenCV library. The robot orients itself by tracking its own movement and the traveled distance. Using these techniques, a system could be developed which is able to rearrange an object in less than one minute.